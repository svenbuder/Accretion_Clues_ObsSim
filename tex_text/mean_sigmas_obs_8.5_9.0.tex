$-0.225 \pm 0.281$%
